% !TeX root = RJwrapper.tex
\title{Geographic Distribution of Williams College Students}
\author{by Jackson Moss '19}

\maketitle

\abstract{%
The \textbf{wgd} package uses publically available data from the
Williams College Registrar to load and determine statistics about the
geographic distribution of the student body at Williams College between
2001 and 2016. The analysis is completed for the 50 states, District of
Columbia, Puerto Rico, and Guam. The package creates various statistics
and graphics illustrating the changes in the geographic distribution of
the student body over the past 15 years.
}

\subsection{Introduction}\label{introduction}

The geographic distribution of the study body, or more simply, where
students are from, is a dynamic that heavily influences the social,
ideological, and academic environment of the college. An understanding
of how this structure changes over time is important to the diversity of
the college and should be watched closely. The package wgd uses string
manipulation on tables extracted from PDFs and organizes the information
in the form of a usable data frame.

The package \textbf{wgd} also creates a variety of statistics and
graphics that illustrate changes in the geographic distribution on a
country, region, and state level. Users can create bar plots showing how
the number of students from a certain state or region changes over time.
These introductory graphics show geographic distribution trends a clear
and concise fashion. The data can also be used to perform more advanced
analysis. For example, external data sets could be used to determine
what variables are associated with changes in the Williams College
geographic distribution.

\subsection{Data}\label{data}

The raw data used for this analysis is in the form of PDFs posted to the
website of the Office of the Registrar of Williams College. The data of
interest are stored in tables inside the PDFs, and are available for the
years between the years 2001 and 2016. In 2011, the Office of the
Registrar began recording the geographic distribution data in a
different way in previous years. Thus, two different methods were
employed to read the data into a usable format, one for the years 2001
to 2010, and one for the years 2011 to 2016.

For 2001 to 2010, the data was converted into text files using
PDFelement, an online PDF to text editor. The result is a character
vector containing two distinct sections. The first x lines reference the
names of the states and the next x lines reference the number of
students from each state. There are many unneeded spaces and periods in
the raw text file. To clean this, spaces and non-alphanumeric characters
are removed.

\begin{Schunk}
\begin{Sinput}
rm_spaces <- gsub(" ", "", frame) 
rm_stuff <- gsub("[[:punct:]]", "", rm_spaces)
\end{Sinput}
\end{Schunk}

Two new character vectors are created, one referring to the name of the
state and one referring to the number of students from that state.
Binding them together creates the format we want.

\begin{Schunk}
\begin{Sinput}
names <- rm_one[nchar(rm_one) > 3] 
numbers <- rm_one[nchar(rm_one) <= 3]
full <- cbind(names, numbers)
\end{Sinput}
\end{Schunk}

For 2011 to 2016, the data can simply be copy and pasted from the PDF
into a text file, and then manually edited to contain a new line for
each state reference. Unneeded spaces are removed, and vectors of the
names of the states and the number of students from each state are
created and bound together.

\begin{Schunk}
\begin{Sinput}
rm_spaces <- gsub(" ", "", data) 
states <- gsub("[0-9]", "", rm_spaces) 
numbers <- gsub("[a-zA-Z]", "", rm_spaces) 
total <- cbind(states, numbers)
\end{Sinput}
\end{Schunk}

These operations yield a two column matrix, one column representing the
names of the states and one column representing the number of students
from each state. However, this 2 column matrix has a variable number of
rows depending on the text file that is being manipulated. In the
instance that a state has zero students from it, the raw data does not
contain that state in the table. Since certain states sometimes have
zero students at Williams, matrices will often contain a different
number of rows and have missing states. In order to report data on all
states between the years of 2001 and 2016, the relevant data must be
added in. A solution is to create a 1 x 2 matrix with the relevant
information and bind it to the data. Several if statements check for
this condition and create the missing data. Do not mind the matrix
created for the else condition, as it is a placeholder and will be
deleted.

\begin{Schunk}
\begin{Sinput}
if(identical(total[total[, 1] %in% c("Nebraska", "Nebraska\t"), 1], character(0)))
\end{Sinput}
\end{Schunk}\begin{Schunk}
\begin{Sinput}
{ a1 <- matrix(c("Nebraska", "0"), 1, 2) } else { a1 <- matrix(1:2, 1, 2) }
\end{Sinput}
\end{Schunk}

Next, the original data and the new matrices are bound together. The
placeholder matrices are removed and the matrix is ordered by
alphabetically by state name. The matrix is coerced into a data frame
that has 53 rows: The 50 states, District of Columbia, Guam, and Puerto
Rico.

\begin{Schunk}
\begin{Sinput}
readEphData("williams2012.txt")[1:4, ]
\end{Sinput}
\begin{Soutput}
#>      state number
#> 1  Alabama      7
#> 2   Alaska     11
#> 3  Arizona      9
#> 4 Arkansas      7
\end{Soutput}
\end{Schunk}

\textbf{Use readEphData}

The information in the locally stored text files can be loaded and
parsed using the readEphData function. The function creates a dataframe
containing the geographic distribution of students for the year of the
user coded text file. Subsequent functions combine the data and format
it into a data frame. The function has one argument and can be used as
follows.

\begin{Schunk}
\begin{Sinput}
readEphData("dataset")
\end{Sinput}
\end{Schunk}

\textbf{Use plot\_state}

The function \textbf{plot\_state} is used to generate graphics that show
the geographic distribution over time for a user-coded state. The
function subsets the data for a particular state and then displays the
information in a bar plot. The function has one argument and is used as
follows.

\begin{Schunk}
\begin{Sinput}
plot_state("State")
\end{Sinput}
\end{Schunk}

The function can display graphics for any of the 50 states, District of
Columbia, Guam, and Puerto Rico. The user should code the state without
spaces and with capital letters where normal. For example:

\begin{Schunk}
\begin{Sinput}
plot_state("Virginia")
plot_state("Florida")
\end{Sinput}

\includegraphics{wgdpdf_files/figure-latex/unnamed-chunk-14-1} \includegraphics{wgdpdf_files/figure-latex/unnamed-chunk-14-2} \end{Schunk}

\textbf{Use plot\_region}

The function \textbf{plot\_region} is used to generate graphics that
show the geographic distribution over time for a user-coded region. The
function subsets the data for a particular region and then displays the
information in a plot. The function has one argument and is used as
follows.

\begin{Schunk}
\begin{Sinput}
plot_region("Region")
\end{Sinput}
\end{Schunk}

The function can display graphics for any of the six regions that this
package defines. The regions and states they contain are as follows:

\begin{itemize}
\tightlist
\item
  West: CA, OR, WA, ID, NV, AZ
\item
  Mountain West: NM, CO, WY, MT, ND, SD, UT
\item
  South: OK, TX, AR, MS, FL, GA, SC, NC, TN, LA, AL
\item
  Midwest: NE, KA, IA, MO, OH, KY, IN, IL, WI, MI, MN
\item
  Northeast: NY, MA, VM, NH, ME, RI, CT
\item
  Mid Atlantic: VA, DE, DC, NJ, WV, PA
\end{itemize}

The user should code the state without spaces and with capital letters
where normal. For example:

\begin{Schunk}
\begin{Sinput}
plot_region("West")
plot_region("Northeast")
\end{Sinput}

\includegraphics{wgdpdf_files/figure-latex/unnamed-chunk-16-1} \includegraphics{wgdpdf_files/figure-latex/unnamed-chunk-16-2} \end{Schunk}

\textbf{Use total\_dist}

The function \textbf{total\_dist} uses the geographic distribution data
from the readEphData function to generate graphics and summary
statistics for user coded options. The function has the argument type
which dictates the type of output generated. The possible options are
Summary, US, and SD. Summary generates a data frame containing the
geographic distribution for the years 2001 to 2016. US generates a bar
plot displaying how total United States enrollment has changed at
Williams from 2001 to 2016. SD generates a data frame containing the
standard deviation in total United States enrollment for the year blocks
2001 to 2010 and 2011 to 2016. The function can be used as follows

\begin{Schunk}
\begin{Sinput}
total_dist("Summary")[1:4, ]
\end{Sinput}
\end{Schunk}

\subsection{Discussion}\label{discussion}

Before delving deeper into the data by state or region, let's look at
the total number of students from the 50 states, District of Columbia,
Guam, and Puerto Rico between 2001 and 2016.

\begin{Schunk}
\begin{Sinput}
total_dist("US")
\end{Sinput}

\includegraphics{wgdpdf_files/figure-latex/unnamed-chunk-18-1} \end{Schunk}

A quick glance at the bar plot shows that the total number has not
changed much in the past 15 years. What is surprising is how flat the
distribution has been over the past six years versus the ten years
before that. Quick calculations of the standard deviation for each time
period yields an interesting result.

\begin{Schunk}
\begin{Sinput}
total_dist("SD")
\end{Sinput}
\begin{Soutput}
#>    2001-2010 2011-2016
#> SD  82.95976  17.02645
\end{Soutput}
\end{Schunk}

The standard deviation of total United States enrollment for the past
six years is almost five times smaller than it was for the preceding ten
years. Did admissions become ``smarter'' about yield calculations
sometime near 2011? What other factors could potentially explain this?

Output from the \textbf{plot\_state} function shows marked upward and
downward trends for different states. California, with only 156 students
in 2001, has become a Williams powerhouse with 267 students in 2016.
Florida has also seen huge gains, rising from 28 in 2001 to 83 in 2016.
Some states have opposite trends. District of Columbia had 26 students
at Williams in 2001 and 2002, but only 12 in 2015 and 2016.
Massachusetts has also seen decreases, from 341 to 280 from 2001 to
2016.

One especially interesting state is Mississippi. The southern state had
an average of 2.4 students at Williams between 2001 and 2010. Not a
single student from Mississippi has matriculated since then.

\begin{Schunk}

\includegraphics{wgdpdf_files/figure-latex/unnamed-chunk-20-1} \end{Schunk}

Output from the \textbf{plot\_region} function shows marked upward and
downward trends for different regions.

\begin{Schunk}

\includegraphics{wgdpdf_files/figure-latex/unnamed-chunk-21-1} \end{Schunk}

A quick glance shows decreases in the Northeast and Mid Atlantic
Regions, and increases in the West and South Regions. The most notable
aspect of the graphic is that the Northeast and West regions have been
converging over the past 15 years.

\subsection{Conclusion}\label{conclusion}

The student geographic distribution data, on both a state and region
level, show marked downward and upward trends. This is an important
result that has no doubt influenced the social, ideological, and
academic environment of the college. A rough statement summarizing the
data would be as follows: Between the years of 2001 and 2016, Williams
has become increasingly represented by students from the West and South
and decreasingly represented by students from the Northeast and Mid
Atlantic.

What is more important than this result are the reasons why these trends
have occurred. Why has the number of Students from Florida increased
almost by a factor of three. Why hasn't a single student from
Mississippi matriculated to Williams since 2010 after an average of 2.4
matriculated the previous ten years? It is unclear whether these trends
are due to internal bias (i.e.~changing admissions goals/quotas for
students from certain states or areas of the country) or the quality of
the students from different areas of the country.

As follows is a quick conceptual model outlining potential reasons why
the geographic distribution might change. This model may guide further
research in this topic.

\textbf{Internal Bias} - Williams desires to admit students from all 50
states for diversity purposes - Williams desires more/less students from
x state or region for y reason

\textbf{Quality of Students} - Williams has become more well known in
areas like the West and South over the past 15 years, leading to more
and better qualified applications. Thus, more students have matriculated
from those areas - The educational quality in state x has changed over
time, resulting in more or less students at Williams from state x

\textbf{Other Reason} - A Williams College admissions counselor in
charge of state or region x is replaced by a counselor who is
better/worse at convincing quality students to apply to Williams, and
thus more/less students from state or region x matriculate to Williams

The actual underpinnings for the geographic distribution of the student
body are almost certainly a combination of the three described above,
and many others. Isolating and compiling data for many of these reasons
would be a difficult task.

An interesting data set to compare with the results from the
\textbf{wgd} package would be data on the admission rates by state. If
the admission rate for a certain state or region remains constant while
the number of students from that state increases or decreases, the
change is likely attributed to the number of students applying from that
area. If the admission rate for a certain state or region trends up or
down, then the change is likely due either the quality of students or
internal bias in Williams admissions. Unfortunately, it is unlikely that
Williams would release this data.

Another interesting data set for comparison would be geographic
distribution data from a school similar to Williams. Under the premise
that students at Williams from the West and South have increased over
the past 15 years due to increased awareness or recognition of Williams,
it would be interesting to see if a similar trend has occurred at
schools with recognition and awareness that has presumably not changed,
or at least not a much. Harvard, for example, has always been well known
and attractive to students all across the country.

Regardless, the package \textbf{wgd} includes functions that allow easy
reading and analysis of geographic distribution data published by
Williams College. Depending on the format of the data, this method can
be generalized to other institutions that publish data in a similar
format. \textbf{wgd} provides a good starting point for analyzing the
geographic distribution of the Williams College student body. As
mentioned above, outside data sets would need to be pulled in order to
analyze the possible underpinnings for the results.

\address{%
Jackson Moss '19\\
Williams College\\
Williamstown, MA, USA\\
}
\href{mailto:jm21@williams.edu}{\nolinkurl{jm21@williams.edu}}

